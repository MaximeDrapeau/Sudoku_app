\documentclass{beamer}
\usetheme{Madrid}
\usecolortheme{beaver}
\usefonttheme{professionalfonts}

\usepackage{fontspec}
\usepackage{xcolor}
\usepackage{tikz}

% === Harmonisation des couleurs ===
\setbeamercolor{structure}{fg=white}
\setbeamercolor{frametitle}{fg=white,bg=blue!70!black}
\setbeamercolor{footline}{fg=white,bg=blue!70!black}
\setbeamercolor{section title}{fg=white,bg=blue!70!black}

% === Page de titre personnalisée ===
\setbeamercolor{title}{fg=black}
\setbeamercolor{subtitle}{fg=black}
\setbeamercolor{author}{fg=black}
\setbeamercolor{institute}{fg=black}
\setbeamercolor{date}{fg=black}

% Utilisation d'un bandeau bleu pour le titre
\setbeamertemplate{title page}{
  \vbox{}
  \vfill
  \begin{centering}
    \colorbox{blue!70!black}{%
      \parbox{\dimexpr\paperwidth-2em}{%
        \centering
        \usebeamerfont{title}\color{white}\inserttitle\par
        \usebeamerfont{subtitle}\color{white}\insertsubtitle%
      }%
    }
    \vspace{1cm}

    {\usebeamerfont{author}\insertauthor\par}
    {\usebeamerfont{institute}\insertinstitute\par}
    {\usebeamerfont{date}\insertdate\par}
  \end{centering}
  \vfill
}

% === Police uniforme ===
\setbeamerfont{frametitle}{family=\sffamily,series=\bfseries,size=\Large}
\setbeamerfont{title}{family=\sffamily,series=\bfseries,size=\LARGE}
\setbeamerfont{author}{family=\sffamily,size=\large}
\setbeamerfont{institute}{family=\sffamily,size=\normalsize}
\setbeamerfont{date}{family=\sffamily,size=\normalsize}
\setbeamerfont{section title}{family=\sffamily,series=\bfseries,size=\huge}
\setbeamerfont{footline}{size=\scriptsize}

% === Pied de page ===
\setbeamertemplate{footline}{%
  \leavevmode%
  \hbox{%
    \begin{beamercolorbox}[wd=.33\paperwidth,ht=2.5ex,dp=1ex,left]{footline}%
      \usebeamerfont{author in head/foot}Maxime, Émile et Mathis%
    \end{beamercolorbox}%
    \begin{beamercolorbox}[wd=.34\paperwidth,ht=2.5ex,dp=1ex,center]{footline}%
      \usebeamerfont{title in head/foot}-- \insertshorttitle{} --%
    \end{beamercolorbox}%
    \begin{beamercolorbox}[wd=.33\paperwidth,ht=2.5ex,dp=1ex,right]{footline}%
      \usebeamerfont{date in head/foot}\textsc{Groupe 51}\hspace{1em}\insertframenumber{} / \inserttotalframenumber\hspace{1em}%
    \end{beamercolorbox}%
  }%
  \vskip0pt%
}

% === Page de section (grand titre au milieu sur bande bleue) ===
\AtBeginSection[]{%
  \begin{frame}[plain]
    \vspace{2.5cm}
    \begin{center}
      \colorbox{blue!70!black}{%
        \parbox{0.9\linewidth}{%
          \centering\color{white}\usebeamerfont{section title}\insertsection%
        }%
      }%
    \end{center}
  \end{frame}
}

% === Métadonnées ===
\title{INM5151 - Projet de Sudoku}
\subtitle{Présentation}
\author{Maxime Drapeau \and Émile Locas \and Mathis Hurtubise}
\institute{Université du Québec à Montréal \\ \textsc{Groupe 51}}
\date{}

\begin{document}

% Page de titre
\begin{frame}
  \titlepage
\end{frame}

% --- Section : Introduction ---
\section{Introduction}

\begin{frame}
  \frametitle{Portée du système}
  Application complète de Sudoku: Complete Sudoku
  \begin{itemize}
    \item Génération et résolution de grilles
    \item Importation de grilles personnalisées
    \item Validation de ces grilles personnalisée
    \item Jeu de Sudoku interactif avec retours en temps réel
  \end{itemize}
\end{frame}

% --- Section : Analyse du système existant ---
\section{Analyse du système existant}
  
\begin{frame}
  \frametitle{Contexte du système actuel}
  \begin{itemize}
    \item Méthode manuel (papier/crayon)
    \item Applications incomplètes
  \end{itemize}
\end{frame}

\begin{frame}
  \frametitle{Description du système actuel}
  Les applications offertes sont fragmentés

  Règles:
  \begin{itemize}
    \item La partie se déroule sur une grille 9x9
    \item Cette grille est aussi séparé en 9 sous-grilles (3x3)
    \item La grille 9x9 est partiellement remplie de chiffres (1-9)
    \item les chiffres (1-9) doivent apparaître une fois par colonne
    \item les chiffres (1-9) doivent apparaître une fois par ligne
    \item les chiffres (1-9) doivent apparaître une fois par sous-grille
  \end{itemize}
\end{frame}

\begin{frame}
  \frametitle{Modes d'opérations}
  Liste non exhaustive de fonctionnalités offertes par les applications sur le marché:
  \begin{itemize}
    \item Résolution automatique
    \item Génération autommatique selon un niveau de difficulté
    \item Jeu intéractif (parfois avec retours en temps réel)
  \end{itemize}
\end{frame}
% --- Section : Les changements envisagés ---
\section{Les changements envisagés}

\begin{frame}
  \frametitle{Objectifs de changement}
  \begin{itemize}
    \item Création d'un système modulaire et performant
    \item Simplifier la génération de grilles valides
    \item Introduire des fonctions d’aide intelligentes
  \end{itemize}
\end{frame}

% --- Section : Le nouveau système proposé ---
\section{Le nouveau système proposé}

\begin{frame}
  \frametitle{Vision générale du système}
  \begin{itemize}
    \item Nouveau moteur basé sur \textbf{Wave Function Collapse (WFC)}
    \item Architecture modulaire et extensible
    \item Composantes principales :
      \begin{itemize}
        \item Moteur WFC (génération et résolution)
        \item Module de validation
        \item Interface utilisateur
      \end{itemize}
  \end{itemize}
\end{frame}

\begin{frame}
  \frametitle{Fonctionnement global}
  \begin{enumerate}
    \item L’utilisateur choisit de résoudre ou générer une grille
    \item Le moteur WFC propage les contraintes
    \item Les possibilités invalides sont éliminées
    \item Le validateur vérifie la cohérence finale
    \item L’interface affiche le résultat au joueur
  \end{enumerate}
  \vspace{0.5cm}
  \textbf{Résultat :} génération rapide, solution garantie et cohérente.
\end{frame}

\begin{frame}
  \frametitle{Architecture du système}
  \centering
  \includegraphics[width=\linewidth,height=0.75\textheight,keepaspectratio]{diagramme4.pdf}
  \vspace{0.3cm}
\end{frame}


\begin{frame}
  \frametitle{Choix technologiques}
  \begin{itemize}
    \item Langage \textbf{Rust} pour la performance et la sécurité mémoire
    \item Architecture modulaire facilitant l’ajout futur de fonctionnalités
    \item Possibilité d’intégrer une interface graphique
  \end{itemize}
\end{frame}

\begin{frame}
  \frametitle{Avantages du système proposé}
  \begin{itemize}
    \item Résolution rapide et fiable des grilles
    \item Génération procédurale cohérente
    \item Architecture claire et maintenable
  \end{itemize}
\end{frame}

\begin{frame}
  \frametitle{Conclusion}
  \begin{itemize}
    \item Le nouveau système transforme le Sudoku en une application intelligente et évolutive
    \item Le moteur WFC apporte robustesse et efficacité
  \end{itemize}
\end{frame}

\end{document}
